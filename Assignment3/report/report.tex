\documentclass{article}
\usepackage[utf8]{inputenc}
\usepackage[portuges]{babel}
\usepackage[a4paper, total={7in, 9in}]{geometry}
\usepackage{graphicx}
\usepackage{float}
\usepackage{fancyvrb}

\newcommand{\question}[1]{
    {\large \textbf{Q: #1}}
    \\
}

\newcommand{\titleRule}{
    \rule{\linewidth}{0.5mm} \\ [0.25cm]
}

\begin{document}

\begin{titlepage}
    \center
    \begin{figure}[H]
        \centering
        \includegraphics[width=4cm]{UM_EENG.jpg}
    \end{figure}
    \textsc{\LARGE Universidade do Minho} \\ [1.5cm]
    \textsc{\Large Mestrado Integrado em Engenharia Informática} \\ [0.5cm]
    \textsc{\large Scripting no Processamento de Linguagem Natural} \\ [0.5cm]

    \titleRule
    {\huge \bfseries RSS Spider}
    \titleRule

    João Pedro Ferreira Vieira A78468 \\
    Miguel Miranda Quaresma A77049 \\[0.25cm]

    \today
\end{titlepage}

\tableofcontents

\newpage

\section{Introdução}
A quantidade de informação existente atualmente torna cada vez mais importante a existência de mecanismos que permitam processar
esta informação e, subsequentemente, procurar por documentos com base em termos. Esta pesquisa deve no entanto, ter em conta a
importância dos termos utilizados no texto, favorecendo nomes a determinantes, sendo útil o uso de algoritmos como TF-IDF que 
têm em conta a frequência de um dado termo no conjunto de documentos. 
A presente ferramenta recorre a estes mecanismos para indexar um conjunto de documentos e permitir ao utilizador realizar pesquisas
nos mesmos.

\section{RSS Spider}
A aplicação desenvolvida apresenta dois modos de funcionamento, o primeiro que consiste em atualizar a base de dados de documentos 
indexados
\subsection{Processamento de documentos}
\subsection{algortimo de pesquisa : TF-IDF}


\end{document}